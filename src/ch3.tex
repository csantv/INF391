% LTeX: language=es
% vi:ft=tex

\chapter{Estado del Arte}

\section{Introducción}

En este capítulo se presenta la revisión sistemática del Estado del Arte
respecto a la gestión de recursos turísticos en sistemas informáticos existentes.

\section{Objetivos de revisión}

En el presente trabajo se utiliza la revisión sistemática con el objetivo de
identificar aquellos estudios, trabajos y productos comerciales los cuales presentan
soluciones para la gestión de la información turística de un país o localidad.
También se busca determinar que clase de servicios adicionales relacionados
con el sector turístico ofrecen para el usuario final.

\section{Preguntas de revisión}

Para lograr los objetivos previamente descritos, se planean las siguientes preguntas
de investigación:

\begin{enumerate}
    \item{Cuáles son los factores y/o riesgos a tomar en cuenta para gestionar
        información de interés para los turistas?}
    \item{De qué manera se han implementado sistemas de información turísticos
        comerciales?}
    \item{Cuáles son los algoritmos implementados en sistemas de información turística
        para recomendar itinerarios a los turistas?}
\end{enumerate}

\section{Estrategia de búsqueda}

\subsection{Motores de búsqueda a usar}

En el presente trabajo se utilizarán los siguientes motores de búsqueda.

\begin{itemize}
    \item{Scopus}
    \item{Springer}
\end{itemize}

\subsection{Cadenas de búsqueda a usar}

\begin{itemize}
    \item{Scopus: \emph{TITLE-ABS-KEY ( tourism  AND recommendation  AND system )  AND  (  LIMIT-TO ( OA ,  "all" ) )  AND  ( LIMIT-TO ( SUBJAREA ,  "COMP" ) ) }}
    \item{Springer: \emph{recommender AND system AND (tourism)}}
\end{itemize}

\subsection{Cantidad de documentos encontrados}

Se presenta a continuación la cantidad de documentos encontrados a la fecha \today{}
usando las cadenas de búsqueda para cada motor.

\begin{table}[h!]
    \centering
    \begin{tabular}{c  c  c}
        \toprule
        Motor de búsqueda & Cantidad de documentos encontrados\\
        \midrule
        Scopus & 176\\
        \midrule
        Springer & 140\\
        \bottomrule
    \end{tabular}
    \caption{Cantidad de resultados obtenidos.}\label{table:cantidad_datos}
\end{table}

\subsection{Definición de criterios de inclusión y exclusión}

\subsubsection{Criterios de inclusión}

\begin{itemize}
    \item{El estudio es un documento de reunión (Conference Paper)}
    \item{El estudio describe el desarrollo de un sistema de información.}
    \item{El estudio describe un algoritmo de recomendaciones turísticas.}
    \item{El estudio está relacionado con turismo y sistemas de información o ciencias de la computación.}
\end{itemize}

\subsubsection{Criterios de exclusión}

\begin{itemize}
    \item{Se debe de realizar un pago para acceder al estudio.}
    \item{El idioma con el que está redactado el estudio es diferente al inglés
        o al español.}
    \item{El estudio no está relacionado con Sistemas de Información o Ciencias de la Computación.}
\end{itemize}


\section{Formulario de extracción de datos}

\newcolumntype{L}[1]{>{\hsize=#1\hsize\raggedright\arraybackslash}X}%
\newcolumntype{R}[1]{>{\hsize=#1\hsize\raggedleft\arraybackslash}X}%
\newcolumntype{C}[1]{>{\hsize=#1\hsize\centering\arraybackslash}X}%

\begin{xltabular}{\textwidth}{C{0.5} L{2} C{0.5}}
    \toprule
    Campo & Descripción & Tipo de pregunta \\
    \midrule
    ID & Identificador o DOI del documento & General \\
    \midrule
    Título & El título del documento & General \\
    \midrule
    Autores & Los autores del documento & General \\
    \midrule
    Año de publicación & El año en el que el documento fue publicado. & General \\
    \midrule
    Método de organización de información & Como se describe en el documento la organización de la información turística en el sistema de información. & P2 \\
    \midrule
    Factores y riesgos & Los factores y riesgos que el documento toma en cuenta respecto a la administración de información turística. & P1 \\
    \midrule
    Soluciones Tecnológicas & Estado del arte de las tecnologías usadas en el sistema descrito por el documento. & P3 \\
    \bottomrule
    \\
    \caption{Formulario de extracción de datos.}\label{table:extracion_datos} \\
\end{xltabular}

