% LTeX: language=es

\documentclass{report}
\usepackage[a4paper, total={7in, 10in}]{geometry}
\usepackage{graphicx}
\usepackage{booktabs}
\usepackage{xltabular}
\usepackage[spanish]{babel}
\usepackage[T1]{fontenc}
\usepackage{hyperref}
\usepackage[es-ES]{datetime2}

\hypersetup{
    colorlinks=true,
    linkcolor=blue,
    filecolor=magenta,
    urlcolor=blue,
    pdftitle={Tesis de Carlos Santos Toro Vera}
}

\graphicspath{ {./res/} }

\begin{document}

% vi:ft=tex
\begin{titlepage}
    \begin{center}
        \vspace*{1cm}
        \Large
        \textbf{PONTIFICIA UNIVERSIDAD CATOLICA DEL PERÚ}

        \vspace{0.5cm}
        \textbf{FACULTAD DE CIENCIAS E INGENIERÍA}

        \vspace{0.5cm}
        \includegraphics[scale=0.8]{pucp}

        \vspace{1.5cm}
        \textbf{Desarrollo de un sistema web para la gestión de contenidos de información turística.}

        \normalsize
        \vspace{3.5cm}
        \textbf{Tesis Para optar por el Título de Ingeniero Informático que presenta el bachiller:}

        \Large
        \vspace{1.5cm}
        \textbf{Carlos Santos Toro Vera}

        \vspace{0.3cm}
        \textbf{20171878}

        \vspace{3cm}
        \textbf{Asesora: Doctora Mariuxi Alexandra Bruzza Moncayo}

        \textbf{Co-asesor: Doctor Manuel Francisco Tupia Anticona}

        \normalsize
        \vspace{3cm}
        {Lima, Marzo de 2023}
    \end{center}
\end{titlepage}



\begin{abstract}

\end{abstract}

\tableofcontents

\cleardoublepage{}
\phantomsection{}
\addcontentsline{toc}{chapter}{\listfigurename}
\listoffigures

\cleardoublepage{}
\phantomsection{}
\addcontentsline{toc}{chapter}{\listtablename}
\listoftables


\chapter{Generalidades}
\section{Problemática}
\subsection{Árbol de Problemas}
\section{Objetivos}
\section{Métodos y Procedimientos}

\chapter{Marco Legal}
\section{Introducción}
\section{Desarrollo del marco}


\chapter{Estado del Arte}

\section{Introducción}

En este capítulo se presenta la revisión sistemática del Estado del Arte
respecto a la gestión de recursos turísticos en sistemas informáticos existentes.

\section{Objetivos de revisión}

En el presente trabajo se utiliza la revisión sistemática con el objetivo de
identificar aquellos estudios, trabajos y productos comerciales los cuales presentan
soluciones para la gestión de la información turística de un país o localidad.
También se busca determinar que clase de servicios adicionales ofrecen para el
usuario final.

\section{Preguntas de revisión}

Para lograr los objetivos previamente descritos, se planean las siguientes preguntas
de investigación:

\begin{itemize}
    \item{De que manera las instituciones publicas gestionan la informacion
        turistica?}
    \item{Cuales son los factores a tomar en cuenta para gestionar informarcion
        de interes para los turistas?}
    \item{}
\end{itemize}
- Casos de implementacion de sistemas informaticos para la gestion turistica

\section{Estrategia de búsqueda}

\subsection{Motores de búsqueda a usar}

En el presente trabajo se utilizaran los siguientes motores de busqueda.

\begin{itemize}
    \item{Scopus}
    \item{IEEE Xplore}
    \item{Springer}
\end{itemize}

\subsection{Cadenas de búsqueda a usar}

TODO

\subsection{Definición de criterios de inclusión y exclusión}

\subsubsection{Criterios de inclusión}

\begin{itemize}
    \item{El estudio describe el desarrollo de un sistema de información.}
    \item{El estudio describe un algoritmo de recomendaciones turísticas.}
    \item{El estudio está relacionado con sistemas de información y turismo.}
\end{itemize}

\subsubsection{Criterios de exclusión}

\begin{itemize}
    \item{Se debe de realizar un pago para acceder al estudio.}
    \item{El idioma con el que está redactado el estudio es diferente al inglés
        o al español.}
    \item{El estudio no está relacionado con Sistemas de Información.}
\end{itemize}


\section{Formulario de extracción de datos}

\newcolumntype{L}[1]{>{\hsize=#1\hsize\raggedright\arraybackslash}X}%
\newcolumntype{R}[1]{>{\hsize=#1\hsize\raggedleft\arraybackslash}X}%
\newcolumntype{C}[1]{>{\hsize=#1\hsize\centering\arraybackslash}X}%

\begin{xltabular}{\textwidth}{C{0.5} L{2} C{0.5}}
    \caption{Formulario de extraccion de datos.}\label{table:extracion_datos} \\
    \toprule
    Campo & Descripcion & Tipo de pregunta \\
    \midrule
    ID & Identificador o DOI del documento & General \\
    \midrule
    Titulo & Titulo del documento & General \\
    \midrule
    Autores & Autores del documento & General \\
    \midrule
    Tipo de fuente
    \midrule
    Año de publicacion
    \bottomrule
\end{xltabular}

\end{document}
