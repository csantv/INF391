\documentclass{article}
\usepackage[a4paper, total={7in, 10in}]{geometry}
\usepackage{blindtext}
\usepackage{graphicx}
\usepackage{fontspec}

%\setmainfont{Arial}
\graphicspath{ {./res/} }

\begin{document}

% vi:ft=tex
\begin{titlepage}
    \begin{center}
        \vspace*{1cm}
        \Large
        \textbf{PONTIFICIA UNIVERSIDAD CATOLICA DEL PERÚ}

        \vspace{0.5cm}
        \textbf{FACULTAD DE CIENCIAS E INGENIERÍA}

        \vspace{0.5cm}
        \includegraphics[scale=0.8]{pucp}

        \vspace{1.5cm}
        \textbf{Desarrollo de un sistema web para la gestión de contenidos de información turística.}

        \normalsize
        \vspace{3.5cm}
        \textbf{Tesis Para optar por el Título de Ingeniero Informático que presenta el bachiller:}

        \Large
        \vspace{1.5cm}
        \textbf{Carlos Santos Toro Vera}

        \vspace{0.3cm}
        \textbf{20171878}

        \vspace{3cm}
        \textbf{Asesora: Doctora Mariuxi Alexandra Bruzza Moncayo}

        \textbf{Co-asesor: Doctor Manuel Francisco Tupia Anticona}

        \normalsize
        \vspace{3cm}
        {Lima, Marzo de 2023}
    \end{center}
\end{titlepage}



\section{Asesores}
\begin{itemize}
    \item{Doctora Mariuxi Bruzza Moncayo}
    \item{Doctor Manuel Francisco Tupia Anticona}
\end{itemize}

\section{Plan de trabajo}

Completar avances/observaciones los lunes, miercoles y jueves de 8pm a 10pm, y
los sabados y domingos de 6pm a 10pm. Entrega de avance los lunes,
miercoles y jueves a las 10 pm.

\subsection{Cronograma de reuniones}

Reuniones semanales los miercoles de 4pm a 5pm.

\section{Área}

Sistemas de informacion

\section{Descripción}

El proyecto consiste en desarrollar un sistema web que permita gestionar
información turística a nivel de un país, tomando como referencia la web:
https://www.visitportugal.com/es.
El sistema va a contar con funcionalidades tales como:

\begin{itemize}
    \item{Información básica para el ingreso al país.}
    \item{Mapas y folletos}
    \item{Enlaces de interés}
    \item{Regiones del país}
    \item{Historia y noticias (incluye regulación vigente de carácter general
        que puede impactar en la visita del turista)

        agentes internaciones, canales de youtube, canales de television

        historia sacar del ministerio de turismo
        }
    \item{
            ¿Qué hacer en el país?

            que proponerle al turista (paquetes de actividades)
        }
    \item{
            Buscardor de actividades, eventos, alojamiento, transporte.

            No es algoritmo muy complicado?

            elaborar paquetes de actividades (tablas para cada ciudad)
            a partir de condiciones iniciales (destino y tipo de actividad)

            formas de transporte, seguridad del transporte

            quieres aparecer en las busquedas de los paquetes? que cada empresa
            pague para inscribirse

            mostrarle todas las alternativas al turista

            solicitar rutas, costos y horarios a las empresas de transporte

            solicitar costos promedio por comida/carta
        }
    \item{
            Imágenes y videos

            relacionado a historia
        }
    \item{Planificación de viajes}
\end{itemize}
Se plantea que el sistema funcione en pantallas tactíles y en equipos `all-in-one'

requerimiento es que sea para ecuador

tactil para instituciones publicas

personas que buscan mucho en los information center, terminales de buses, museos
paraderos oficiales de taxis.

\subsection{¿Qué problema busca resolver?}

administrados por el gobierno

ingreso de datos puede ser un sistema comun y corriente.

para el cliente puede ser una app o una pagina web.

generar propio teclado vs teclado del sistema operativo

Generacion de paquetes con informacion unificada. Evitar el uso de multiples
paginas para obtener informacion.

Como guardar la informacion para uso futuro. (correo, whatsapp, impresion)

Las agencias turisticas cobran. Si no tienen proveedores en ciertos lugares no
es posible que estos provean informacion, hostales, restaurantes, turismo.

\subsection{¿Qué resultado espera lograr?}

un gestor de contenido que maneje informacion turistica, gastronomica, comercial
(restaurantes, centros turisticos, parques de diversiones).

atracciones locales controladas por instituciones publicas.

\subsection{¿Qué métodos y procedimientos espera usar?}

sacar informacion de migraciones de cada departamento/canton. Con conexion a maps.

o mapas elaborados por instituciones publicas.

enlaces de interes.

\subsection{Alcance}



\end{document}
