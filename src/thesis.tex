\documentclass{article}
\usepackage[a4paper, total={7in, 10in}]{geometry}
\usepackage{blindtext}
\usepackage{graphicx}
\usepackage{fontspec}
\usepackage{booktabs}
\usepackage[utf8]{inputenc}
\usepackage{hyperref}

\hypersetup{
    colorlinks=true,
    linkcolor=blue,
    filecolor=magenta,
    urlcolor=blue,
    pdftitle={My Resume}
}


%\setmainfont{Arial}
\graphicspath{ {./res/} }

\begin{document}

% vi:ft=tex
\begin{titlepage}
    \begin{center}
        \vspace*{1cm}
        \Large
        \textbf{PONTIFICIA UNIVERSIDAD CATOLICA DEL PERÚ}

        \vspace{0.5cm}
        \textbf{FACULTAD DE CIENCIAS E INGENIERÍA}

        \vspace{0.5cm}
        \includegraphics[scale=0.8]{pucp}

        \vspace{1.5cm}
        \textbf{Desarrollo de un sistema web para la gestión de contenidos de información turística.}

        \normalsize
        \vspace{3.5cm}
        \textbf{Tesis Para optar por el Título de Ingeniero Informático que presenta el bachiller:}

        \Large
        \vspace{1.5cm}
        \textbf{Carlos Santos Toro Vera}

        \vspace{0.3cm}
        \textbf{20171878}

        \vspace{3cm}
        \textbf{Asesora: Doctora Mariuxi Alexandra Bruzza Moncayo}

        \textbf{Co-asesor: Doctor Manuel Francisco Tupia Anticona}

        \normalsize
        \vspace{3cm}
        {Lima, Marzo de 2023}
    \end{center}
\end{titlepage}



\section{Área}
Sistemas de información

\section{Asesores}
\begin{itemize}
    \item{Doctora Mariuxi Alexandra Bruzza Moncayo}
    \item{Doctor Manuel Francisco Tupia Anticona}
\end{itemize}

\section{Plan de trabajo}

Completar avances y levanto de observaciones los lunes, miercoles y jueves de
8pm a 10pm, y los sabados y domingos de 6pm a 10pm.
Se entregaran avances los lunes, miercoles y jueves a las 10 pm.

\subsection{Cronograma de reuniones}

\begin{center}
    \begin{tabular}{c c c c}
        \toprule
        Semana & Tema & Fecha & Horario \\
        \midrule
        1 & Revisión de entregable EP1.1. & 22/03/2023 & 16:00--17:00 \\
        \midrule
        2 & Revisión de entregable EP1.2. & 29/03/2023 & 16:00--17:00 \\
        \midrule
        3 & Revisión de entregable EP1.3. & 05/04/2023 & 16:00--17:00 \\
        \midrule
        4 & Revisión de entregable EP1.4. & 12/04/2023 & 16:00--17:00 \\
        \midrule
        5 & Revisión de entregable EP1.5. & 19/04/2023 & 16:00--17:00 \\
        \midrule
        6 & Revisión de entregable EP1. & 26/04/2023 & 16:00--17:00 \\
        \midrule
        7 & Revisión de entregable EP2.1. & 03/05/2023 & 16:00--17:00 \\
        \midrule
        8 & Revisión de avance para E2. & 10/05/2023 & 16:00--17:00 \\
        \midrule
        9 & No hay reunion & --- & --- \\
        \midrule
        10 & Revisión de entregable E2. & 24/05/2023 & 16:00--17:00 \\
        \midrule
        11 & Revisión de avance para E3. & 31/05/2023 & 16:00--17:00 \\
        \midrule
        12 & Revisión de entregable E3. & 07/06/2023 & 16:00--17:00 \\
        \midrule
        13 & Revisión de entregable E4. & 14/06/2023 & 16:00--17:00 \\
        \bottomrule
    \end{tabular}
\end{center}

\section{Descripción}

El proyecto consiste en desarrollar un sistema web que permita gestionar
información turística a nivel de un país, tomando como referencia la web:
\url{https://www.visitportugal.com/es}. Es verdad que la mayoria de instituciones
gubernamentales en el mundo y empresas turisticas locales ya poseen informacion
para turistas, pero esta informacion tiende a ser poco interactiva o no existente
en los casos en que la empresa turistica no posee convenios con empresas locales
en las ubicaciones que un turista este interesado.\\
Esta tesis propone una solucion de software la cual remediara los problemas
anteriormente mencionados, ademas de incluir funcionalidades adicionales tales como:
\begin{itemize}
    \item{Mostrar información básica para el ingreso al país.}
    \item{Mostrar enlaces de interés}
    \item{Mostrar historia del lugar y noticias actuales.}
    \item{Buscardor de actividades, eventos, alojamiento, transporte.}
    \item{Planificación de viajes}
    \item{Entre otros\ldots{}}
\end{itemize}
Gran parte de la informacion requerida se obtendra utilizando fuentes públicas
de cada gobierno local. Esta informacion será ingresada utilizando un sistema
simple de administracion de contenido incluido en el sistema. Esta informacion
se mostrara a los usuarios utilizando una aplicacion web la cual tendra funcionalidades
adicionales las cuales le permitan trabajar en dispositivos tactiles y `All-in-one'.
Adicionalmente, se implementara un algoritmo propietario el cual se encargara
de la generacion de paquetes de viajes (lista de actividades, informacion de transporte,
informacion adicional, etc).

\end{document}
