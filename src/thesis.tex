\documentclass{article}
\usepackage[a4paper, total={7in, 10in}]{geometry}
\usepackage{blindtext}
\usepackage{graphicx}
%\usepackage{fontspec}
%\setmainfont{Arial}

\graphicspath{ {./res/} }

\begin{document}

% vi:ft=tex
\begin{titlepage}
    \begin{center}
        \vspace*{1cm}
        \Large
        \textbf{PONTIFICIA UNIVERSIDAD CATOLICA DEL PERÚ}

        \vspace{0.5cm}
        \textbf{FACULTAD DE CIENCIAS E INGENIERÍA}

        \vspace{0.5cm}
        \includegraphics[scale=0.8]{pucp}

        \vspace{1.5cm}
        \textbf{Desarrollo de un sistema web para la gestión de contenidos de información turística.}

        \normalsize
        \vspace{3.5cm}
        \textbf{Tesis Para optar por el Título de Ingeniero Informático que presenta el bachiller:}

        \Large
        \vspace{1.5cm}
        \textbf{Carlos Santos Toro Vera}

        \vspace{0.3cm}
        \textbf{20171878}

        \vspace{3cm}
        \textbf{Asesora: Doctora Mariuxi Alexandra Bruzza Moncayo}

        \textbf{Co-asesor: Doctor Manuel Francisco Tupia Anticona}

        \normalsize
        \vspace{3cm}
        {Lima, Marzo de 2023}
    \end{center}
\end{titlepage}



\section{Tema propuesto}
    El proyecto consiste en desarrollar un sistema web que permita gestionar información turística a nivel de un país, tomando como referencia la web: https://www.visitportugal.com/es. El sistema va a contar con funcionalidades tales como:
Información básica para el ingreso al país
Mapas y folletos
Enlaces de interés
Regiones del país
Historia y noticias (incluye regulación vigente de carácter general que puede impactar en la visita del turista)
¿Qué hacer en el país?
Buscardor de actividades, eventos, alojamiento, transporte
Imágenes y videos
Planificación de viajes
    Se plantea que el sistema funcione en pantallas tactíles y en equipos all-in-one

\section{Observaciones y entregables de la investigación}

Análisis de la solución

Diseño

Implementación

Pruebas

Caso de estudio aplicado al Ecuador, provincia de Manabí

Consideración: que el sistema funcione para tecnología de pantallas `touch screen'


\end{document}
